%%
%% Copyright (c) 2018-2019 Weitian LI <wt@liwt.net>
%% CC BY 4.0 License
%%
%% Created: 2018-04-11
%%

% !TEX program = xelatex

% Chinese version
\documentclass[zh]{resume}

% Adjust icon size (default: same size as the text)
\iconsize{\Large}

% File information shown at the footer of the last page
\fileinfo{%
  \faCopyright{} 2023, Yipeng Liu \hspace{0.5em}
  \creativecommons{by}{4.0} \hspace{0.5em}
  \githublink{lwpie}{resume} \hspace{0.5em}
  \faEdit{} \today
}

\name{一芃}{刘}

\keywords{Linux, Python, C, Shell, DevOps, SysAdmin}

% \tagline{\icon{\faBinoculars}} <position-to-look-for>}
% \tagline{<current-position>}

% \photo{<height>}{<filename>}

\profile{
  \mobile{15::24652}
  \email{liu@yipe.ng}
  \github{lwpie} \\
  \university{清华大学 \textbullet 交叉信息研究院}
  % \degree{博士在读}
  \birthday{2002-04}
  \home{上海}
  % Custom information:
  % \icontext{<icon>}{<text>}
  % \iconlink{<icon>}{<link>}{<text>}
}

\begin{document}
\makeheader

%======================================================================
% Summary & Objectives
%======================================================================
{\onehalfspacing\vspace{0.5em}\hspace{2em}%
交叉信息研究院博士一年级在读,
计算机科学与技术系本科优秀毕业生,
现任学生网络与开源软件协会(\link{https://tuna.moe}{\texttt{TUNA}})会长。
热爱计算机专业,熟悉 \texttt{C++,Python} 等编程语言与 \texttt{GNU/Linux} 的使用,具有一定英文基础。
完成多个挑战性课程项目并参与校级项目,在软硬件领域、工程与科研方向均进行了探索与尝试,
取得全国特等奖在内的多项荣誉,并曾入围系综合奖学金答辩。
实践开源精神,参与开源社区活动并积极开放源代码;
加入答疑坊等志愿组织,累计志愿服务时长超五百小时。
公路骑行与游泳爱好者,现役交叉信息研究院泳队。
\par}

%======================================================================
\sectionTitle{技能和语言}{\faWrench}
%======================================================================
\begin{competences}
  \separator{0.1em}

  \comptence{编程}{%
    \texttt{C/C++,Python,Rust}
  }
  \comptence{}{%
    \texttt{SQL,Shell,Web}
  }

  \comptence{\icon{\faLanguage} 英语}{%
    上海英语高考137分
  }
  \comptence{}{%
    大学英语四级考试647分 / 六级考试623分
  }
\end{competences}

%======================================================================
\sectionTitle{科研经历}{\faGraduationCap}
%======================================================================
\begin{educations}
  \separator{0.2em}

  \education%
    {2021.06}%
    [现在]%
    {交叉信息研究院}%
    {张焕晨}%
    {数据库}%
    {云数据仓库存储格式与自动聚合}

  \separator{0.5em}
  \education%
    {2020.02}%
    [2020.10]%
    {计算机科学与技术系}%
    {刘知远}%
    {自然语言处理}%
    {疫情谣言传播结构分析}
\end{educations}

%======================================================================
\sectionTitle{主要项目与荣誉}{\faBriefcase}
%======================================================================
\begin{awards}
  \separator{0.2em}

  \award%
    {2022.06}%
    {计算机系优秀毕业生}%
    {酒井之星}
  \separator{0.5em}

  \study%
    {2021.08}%
    [2021.04]%
    {全国大学生计算机系统能力培养大赛}%
    {(\texttt{C++,ARM})}%
    {编译系统设计赛特等奖,并获华为毕昇杯}
  \separator{0.5em}

  \study%
    {2021.05}%
    [2020.10]%
    {NaiveDB++}%
    {(\texttt{C++,SQL})}%
    {支持持久化序列存储、索引加速、JOIN操作、MVCC与WAL故障恢复的数据库框架}
  \separator{0.5em}

  \award%
    {2020.10}%
    {计算机系单项奖学金}%
    {学业进步、社会工作、科技创新、志愿公益}
  \separator{0.5em}

  \study%
    {2021.01}%
    [2020.09]%
    {Untitled Router}%
    {(\texttt{Verilog,RISC-V,C})}%
    {在 FPGA 上实现五级流水线 RISC-V 处理器,并基于此实现支持 RIPv2 协议的硬件转发引擎}
  \separator{0.5em}

  \study%
    {2020.04}%
    [2020.03]%
    {109周年校庆云合影}%
    {(\texttt{Flask,Vue})}%
    {访问量达百余万,收到多封知名校友感谢信}
  \separator{0.5em}

  \study%
    {2019.12}%
    [2019.10]%
    {清华大学第二十一届电子设计大赛}%
    {(\texttt{STM32,C})}%
    {基于传感器的智能车对战积分赛,商汤特等奖}
  \separator{0.5em}

  \award%
    {2019.04}%
    {清华大学第三十七届挑战杯}%
    {交叉学科二等奖、发明专利一项}
  \separator{0.5em}
  
  \award%
    {2017.05}%
    {国际数学建模挑战赛}%
    {大中华区特等奖 Outstanding Winner}
  \separator{0.5em}

  \award%
    {2016.04}%
    {美国大学生数学建模竞赛}%
    {一等奖 Meritorious Winner}

\end{awards}

\end{document}
